\PassOptionsToPackage{unicode=true}{hyperref} % options for packages loaded elsewhere
\PassOptionsToPackage{hyphens}{url}
%
\documentclass[]{article}
\usepackage{lmodern}
\usepackage{amssymb,amsmath}
\usepackage{ifxetex,ifluatex}
\usepackage{fixltx2e} % provides \textsubscript
\ifnum 0\ifxetex 1\fi\ifluatex 1\fi=0 % if pdftex
  \usepackage[T1]{fontenc}
  \usepackage[utf8]{inputenc}
  \usepackage{textcomp} % provides euro and other symbols
\else % if luatex or xelatex
  \usepackage{unicode-math}
  \defaultfontfeatures{Ligatures=TeX,Scale=MatchLowercase}
\fi
% use upquote if available, for straight quotes in verbatim environments
\IfFileExists{upquote.sty}{\usepackage{upquote}}{}
% use microtype if available
\IfFileExists{microtype.sty}{%
\usepackage[]{microtype}
\UseMicrotypeSet[protrusion]{basicmath} % disable protrusion for tt fonts
}{}
\IfFileExists{parskip.sty}{%
\usepackage{parskip}
}{% else
\setlength{\parindent}{0pt}
\setlength{\parskip}{6pt plus 2pt minus 1pt}
}
\usepackage{hyperref}
\hypersetup{
            pdfborder={0 0 0},
            breaklinks=true}
\urlstyle{same}  % don't use monospace font for urls
\usepackage{longtable,booktabs}
% Fix footnotes in tables (requires footnote package)
\IfFileExists{footnote.sty}{\usepackage{footnote}\makesavenoteenv{longtable}}{}
\setlength{\emergencystretch}{3em}  % prevent overfull lines
\providecommand{\tightlist}{%
  \setlength{\itemsep}{0pt}\setlength{\parskip}{0pt}}
\setcounter{secnumdepth}{0}
% Redefines (sub)paragraphs to behave more like sections
\ifx\paragraph\undefined\else
\let\oldparagraph\paragraph
\renewcommand{\paragraph}[1]{\oldparagraph{#1}\mbox{}}
\fi
\ifx\subparagraph\undefined\else
\let\oldsubparagraph\subparagraph
\renewcommand{\subparagraph}[1]{\oldsubparagraph{#1}\mbox{}}
\fi

% set default figure placement to htbp
\makeatletter
\def\fps@figure{htbp}
\makeatother


\date{}

\begin{document}

\hypertarget{watchdoge}{%
\section{WatchDoge}\label{watchdoge}}

\hypertarget{documento-di-analisi-e-specifica}{%
\subsection{Documento di Analisi e
Specifica}\label{documento-di-analisi-e-specifica}}

\hypertarget{versione-1.0---2-novembre-2018}{%
\subsubsection{Versione 1.0 - 2 novembre
2018}\label{versione-1.0---2-novembre-2018}}

\textbf{Team DogeDroid} - Dario Lazzaro - Giovanni Scodeller - Giulio
Zausa - Samuele Casarin - Sandro Baccega

{[}TOC{]}

\hypertarget{introduzione}{%
\section{1 Introduzione}\label{introduzione}}

\hypertarget{funzionalituxe0-generali-del-sistema}{%
\subsection{1.1 Funzionalità Generali del
Sistema}\label{funzionalituxe0-generali-del-sistema}}

L'obbiettivo del progetto è lo sviluppo di un sistema antifurto
destinato all'uso domestico. Tale sistema sarà composto principalmente
da due sottosistemi: - un robot che, quando abilitato, rileva le
intrusioni all'interno dell'ambiente domestico (entro certi limiti) e le
invia all'applicazione Android; - un'applicazione Android che
abilita/disabilita il robot su decisione dell'utente e notifica l'utente
riguardo le segnalazioni ricevute dal robot.

\hypertarget{descrizione-del-documento}{%
\subsection{1.2 Descrizione del
Documento}\label{descrizione-del-documento}}

\hypertarget{da-cambiare-la-descrizione-dei-campi}{%
\subsubsection{Da cambiare la descrizione dei
campi}\label{da-cambiare-la-descrizione-dei-campi}}

\begin{figure}[htbp]
\centering
\begin{tabular}{|c|l|}
\hline

\textbf{Nome}
 & 
Nome del requisito funzionale
\\

\textbf{ID}
 & 
Codice univoco del requisito funzionale
\\

\textbf{Descrizione}
 & 
Descrizione del requisito funzionale
\\

\textbf{Motivazione}
 & 
Descrive il perché abbiamo deciso di includere questo requisito
funzionale
\\

\textbf{Influisce}
 & 
Indica se questo requisito funzionale influisce sul sistema
\\

\textbf{Specifica}
 & 
Indica la specifica dei requisiti funzionali
\\
\hline
\end{tabular}
\end{figure}

\hypertarget{glossario}{%
\section{2 Glossario}\label{glossario}}

\hypertarget{watchdoge-1}{%
\section{3 WatchDoge}\label{watchdoge-1}}

\hypertarget{modelli-del-sistema}{%
\subsection{3.1 Modelli del Sistema}\label{modelli-del-sistema}}

\begin{figure}[htbp]
\centering
\begin{tabular}{|c|l|}
\hline

\textbf{Codice}
 & 
UC-01
\\

\textbf{Nome}
 & 
Abilita Accoppiamento
\\

\textbf{Scopo}
 & 
Associare il robot ad un dispositivo Android per permetterne il
controllo all'utente.
\\

\textbf{Attori}
 & 
Utente
\\

\textbf{Precondizioni}
 & 
-
\\

\textbf{Trigger}
 & 
Pressione del pulsante \emph{Abilita accoppiamento} (robot)
\\

\textbf{Descrizione}
 & 
Il robot accetta richieste di accoppiamento per un certo tempo.
\\

\textbf{Alternative}
 & 
-
\\

\textbf{Postcondizioni}
 & 
Il robot è accoppiato con un dispositivo Android, oppure non è
accoppiato
\\
\hline
\end{tabular}
\end{figure}

\begin{figure}[htbp]
\centering
\begin{tabular}{|c|l|}
\hline

\textbf{Codice}
 & 
UC-02
\\

\textbf{Nome}
 & 
Visualizza Codice per Accoppiamento
\\

\textbf{Scopo}
 & 
Mitigare la presa di controllo indesiderata del robot da parte di
qualcun altro diverso dall'utente.
\\

\textbf{Attori}
 & 
Utente
\\

\textbf{Precondizioni}
 & 
UC-01
\\

\textbf{Trigger}
 & 
Il robot riceve una richiesta di accoppiamento.
\\

\textbf{Descrizione}
 & 
Il robot visualizza un codice con cui il dispositivo che ha inviato la
richiesta di accoppiamento deve rispondere per poter completare la
procedura di accoppiamento.
\\

\textbf{Alternative}
 & 
-
\\

\textbf{Postcondizioni}
 & 
Il successo della verifica garantisce che il dispositivo che ha inviato
la richiesta di accoppiamento è effettivamente quello dell'utente.
\\
\hline
\end{tabular}
\end{figure}

\begin{figure}[htbp]
\centering
\begin{tabular}{|c|l|}
\hline

\textbf{Codice}
 & 
UC-03
\\

\textbf{Nome}
 & 
Accoppia Robot
\\

\textbf{Scopo}
 & 
Associare il dispositivo Android al robot per permetterne il controllo
all'utente.
\\

\textbf{Attori}
 & 
Utente
\\

\textbf{Precondizioni}
 & 
Connessione alla stessa rete del robot, UC-01
\\

\textbf{Trigger}
 & 
Primo avvio dell'applicazione Android e pressione del pulsante
\emph{Accoppia robot} (applicazione Android)
\\

\textbf{Descrizione}
 & 
L'applicazione Android cerca robot presenti nella stessa rete dello
smartphone entro un certo tempo. Al termine, l'applicazione Android
visualizza la lista di robot trovati, permettendo all'utente di
selezionare quello da associare.
\\

\textbf{Alternative}
 & 
-
\\

\textbf{Postcondizioni}
 & 
Il dispositivo Android è accoppiato con il robot
\\
\hline
\end{tabular}
\end{figure}

\begin{figure}[htbp]
\centering
\begin{tabular}{|c|l|}
\hline

\textbf{Codice}
 & 
UC-04
\\

\textbf{Nome}
 & 
Inserisci Codice per Accoppiamento
\\

\textbf{Scopo}
 & 
Mitigare la presa di controllo indesiderata del robot da parte di
qualcun altro diverso dall'utente.
\\

\textbf{Attori}
 & 
Utente
\\

\textbf{Precondizioni}
 & 
UC-03
\\

\textbf{Trigger}
 & 
Selezione di un robot da associare
\\

\textbf{Descrizione}
 & 
L'utente inserisce il codice di accoppiamento visualizzato dal robot e
lo inserisce in un apposito campo. Successivamente, l'applicazione
Android invia il codice al robot per la verifica.
\\

\textbf{Alternative}
 & 
-
\\

\textbf{Postcondizioni}
 & 
Il successo della verifica garantisce che il dispositivo che ha inviato
la richiesta di accoppiamento al robot è effettivamente quello
dell'utente.
\\
\hline
\end{tabular}
\end{figure}

\begin{figure}[htbp]
\centering
\begin{tabular}{|c|l|}
\hline

\textbf{Codice}
 & 
UC-05
\\

\textbf{Nome}
 & 
Visualizza Stato Robot
\\

\textbf{Scopo}
 & 
Consentire all'utente di visualizzare in tempo reale le informazioni
relative al robot.
\\

\textbf{Attori}
 & 
Utente
\\

\textbf{Precondizioni}
 & 
UC-03
\\

\textbf{Trigger}
 & 
Aprire l'applicazione Android
\\

\textbf{Descrizione}
 & 
L'utente visualizza in tempo reale se l'antifurto è abilitato oppure no
e la sua programmazione attraverso un'apposita interfaccia
grafica.
\\

\textbf{Alternative}
 & 
-
\\

\textbf{Postcondizioni}
 & 
L'utente è al corrente dello stato del robot.
\\
\hline
\end{tabular}
\end{figure}

\begin{figure}[htbp]
\centering
\begin{tabular}{|c|l|}
\hline

\textbf{Codice}
 & 
UC-06
\\

\textbf{Nome}
 & 
Abilita/Disabita Robot
\\

\textbf{Scopo}
 & 
Consentire all'utente di attivare/disattivare la funzione di antifurto
del robot.
\\

\textbf{Attori}
 & 
Utente
\\

\textbf{Precondizioni}
 & 
UC-03
\\

\textbf{Trigger}
 & 
Pressione del pulsante \emph{ON/OFF} (applicazione Android)
\\

\textbf{Descrizione}
 & 
L'utente cambia lo stato del robot ad attivo/disattivo.
\\

\textbf{Alternative}
 & 
-
\\

\textbf{Postcondizioni}
 & 
Il robot è attivo/disattivo.
\\
\hline
\end{tabular}
\end{figure}

\begin{figure}[htbp]
\centering
\begin{tabular}{|c|l|}
\hline

\textbf{Codice}
 & 
UC-07
\\

\textbf{Nome}
 & 
Imposta Programmazione
\\

\textbf{Scopo}
 & 
Consentire all'utente di decidere quando il robot deve
attivarsi/disattivarsi automaticamente in determinate fasce
orarie.
\\

\textbf{Attori}
 & 
Utente
\\

\textbf{Precondizioni}
 & 
UC-03
\\

\textbf{Trigger}
 & 
Pressione del pulsante \emph{Programma}
\\

\textbf{Descrizione}
 & 
L'utente abilita/disabilita la modalità automatica attraverso un
interrutore e imposta le fasce orarie di attivazione del robot
attraverso una tabella oraria settimanale.
\\

\textbf{Alternative}
 & 
-
\\

\textbf{Postcondizioni}
 & 
Il robot è configurato in modalità automatica per attivarsi negli orari
stabiliti dall'utente.
\\
\hline
\end{tabular}
\end{figure}

\begin{figure}[htbp]
\centering
\begin{tabular}{|c|l|}
\hline

\textbf{Codice}
 & 
UC-08
\\

\textbf{Nome}
 & 
Controllo Remoto
\\

\textbf{Scopo}
 & 
Consentire all'utente di guardare in tempo reale le immagini catturate
dal robot e spostare la visuale.
\\

\textbf{Attori}
 & 
Utente
\\

\textbf{Precondizioni}
 & 
UC-03
\\

\textbf{Trigger}
 & 
Pressione del pulsante \emph{Controllo remoto}
\\

\textbf{Descrizione}
 & 
L'utente può connettersi da remoto al robot e visualizza le immagini
catturate dalla fotocamera incorporata; inoltre, attraverso dei pulsanti
direzionali, può spostare la visuale della fotocamera in base all'angolo
di rotazione.
\\

\textbf{Alternative}
 & 
-
\\

\textbf{Postcondizioni}
 & 
L'utente acquisisce una panoramica istantanea dell'ambiente dove è
disposto il robot.
\\
\hline
\end{tabular}
\end{figure}

\begin{figure}[htbp]
\centering
\begin{tabular}{|c|l|}
\hline

\textbf{Codice}
 & 
UC-09
\\

\textbf{Nome}
 & 
Cambia Stato Robot
\\

\textbf{Scopo}
 & 
Attivare/disattivare/configurare una o più funzioni del robot.
\\

\textbf{Attori}
 & 
Utente
\\

\textbf{Precondizioni}
 & 
UC-03
\\

\textbf{Trigger}
 & 
Una richiesta proveniente dall'applicazione Android
\\

\textbf{Descrizione}
 & 
Attraverso l'interfaccia grafica, l'utente invia una richiesta al robot
per cambiare il suo stato.
\\

\textbf{Alternative}
 & 
-
\\

\textbf{Postcondizioni}
 & 
Lo stato del robot viene cambiato a seconda della richiesta
inviata.
\\
\hline
\end{tabular}
\end{figure}

\begin{figure}[htbp]
\centering
\begin{tabular}{|c|l|}
\hline

\textbf{Codice}
 & 
UC-10
\\

\textbf{Nome}
 & 
Ottieni Stato Robot
\\

\textbf{Scopo}
 & 
Consentire all'applicazione Android di riportare le informazioni
relative alle funzioni del robot.
\\

\textbf{Attori}
 & 
Utente
\\

\textbf{Precondizioni}
 & 
UC-03
\\

\textbf{Trigger}
 & 
Una richiesta proveniente dall'applicazione Android
\\

\textbf{Descrizione}
 & 
L'applicazione Android invia una richiesta al robot per ottenere il suo
stato.
\\

\textbf{Alternative}
 & 
-
\\

\textbf{Postcondizioni}
 & 
L'applicazione Android ottiene lo stato del robot a seconda della
richiesta inviata.
\\
\hline
\end{tabular}
\end{figure}

\begin{figure}[htbp]
\centering
\begin{tabular}{|c|l|}
\hline

\textbf{Codice}
 & 
UC-11
\\

\textbf{Nome}
 & 
Visualizza Storico Segnalazioni
\\

\textbf{Scopo}
 & 
Consentire all'utente di avere un archivio delle segnalazioni di
intrusioni rilevate dal robot.
\\

\textbf{Attori}
 & 
Utente
\\

\textbf{Precondizioni}
 & 
UC-03
\\

\textbf{Trigger}
 & 
Pressione del pulsante \emph{Storico} (applicazione Android)
\\

\textbf{Descrizione}
 & 
L'applicazione Android visualizza la lista delle ultime
segnalazioni.
\\

\textbf{Alternative}
 & 
-
\\

\textbf{Postcondizioni}
 & 
L'utente ha una visione generale delle segnalazioni rilevate dal
robot.
\\
\hline
\end{tabular}
\end{figure}

\begin{figure}[htbp]
\centering
\begin{tabular}{|c|l|}
\hline

\textbf{Codice}
 & 
UC-12
\\

\textbf{Nome}
 & 
Visualizza Notifica Segnalazione
\\

\textbf{Scopo}
 & 
Consentire all'utente di ricevere immediatamente una segnalazione di
intrusione.
\\

\textbf{Attori}
 & 
Utente
\\

\textbf{Precondizioni}
 & 
UC-03
\\

\textbf{Trigger}
 & 
Ricezione di una segnalazione dal robot
\\

\textbf{Descrizione}
 & 
L'applicazione Android mostra una notifica con lo scopo di avvertire
l'utente di un'intrusione rilevata dal robot. Inoltre, alla notifica
viene allegata un'immagine dell'intruso.
\\

\textbf{Alternative}
 & 
-
\\

\textbf{Postcondizioni}
 & 
L'utente può essere informato immediatamente quando viene rilevata
un'intrusione.
\\
\hline
\end{tabular}
\end{figure}

\begin{figure}[htbp]
\centering
\begin{tabular}{|c|l|}
\hline

\textbf{Codice}
 & 
UC-13
\\

\textbf{Nome}
 & 
Innesca Antifurto
\\

\textbf{Scopo}
 & 
Mitigare le intrusioni all'interno dell'ambiente del robot.
\\

\textbf{Attori}
 & 
Intruso
\\

\textbf{Precondizioni}
 & 
UC-03
\\

\textbf{Trigger}
 & 
La fotocamera del robot rileva del movimento all'interno
dell'ambiente
\\

\textbf{Descrizione}
 & 
Il robot invia una segnalazione di intrusione al dispositivo accoppiato
ed emette un suono d'allarme.
\\

\textbf{Alternative}
 & 
-
\\

\textbf{Postcondizioni}
 & 
L'intruso viene rilevato e la sua presenza viene segnalata
all'utente.
\\
\hline
\end{tabular}
\end{figure}

\begin{figure}[htbp]
\centering
\begin{tabular}{|c|l|}
\hline

\textbf{Codice}
 & 
UC-14
\\

\textbf{Nome}
 & 
Scatta Fotografia Intruso
\\

\textbf{Scopo}
 & 
Fornire all'utente un'immagine dell'intruso per tentare la sua
identificazione.
\\

\textbf{Attori}
 & 
Utente
\\

\textbf{Precondizioni}
 & 
UC-13
\\

\textbf{Trigger}
 & 
La fotocamera del robot rileva del movimento all'interno
dell'ambiente
\\

\textbf{Descrizione}
 & 
Il robot invia un immagine dalla fotocamera puntata nella direzione in
cui ha rilevato del movimento, dove probabilmente è presente
l'intruso.
\\

\textbf{Alternative}
 & 
-
\\

\textbf{Postcondizioni}
 & 
L'utente potrebbe avere una prova per identificare l'intruso.
\\
\hline
\end{tabular}
\end{figure}

\hypertarget{definizione-dei-requisiti-funzionali}{%
\subsection{3.2 Definizione dei Requisiti
Funzionali}\label{definizione-dei-requisiti-funzionali}}

\hypertarget{robot}{%
\subsubsection{ROBOT}\label{robot}}

\begin{figure}[htbp]
\centering
\begin{tabular}{|c|l|}
\hline

\textbf{Nome}
 & 
Accoppiamento ROBOT
\\

\textbf{ID}
 & 
FR1
\\

\textbf{Descrizione}
 & 
Si collega il dispositivo ROBOT allo smartphone del proprietario
\\

\textbf{Motivazione}
 & 
Possibilità di visualizzare informazioni e comandare l'apparato a
distanza, grazie ad una rete di collegamento
\\

\textbf{Influisce}
 & 
Funzionalità principale del sistema d'allarme
\\

\textbf{Specifica}
 & 
FRS1
\\
\hline
\end{tabular}
\end{figure}

~

\begin{figure}[htbp]
\centering
\begin{tabular}{|c|l|}
\hline

\textbf{Nome}
 & 
Rilevamento Intrusi
\\

\textbf{ID}
 & 
FR2
\\

\textbf{Descrizione}
 & 
Se dovesse avvenire un movimento nell'ambiente il dispositivo Lego se ne
accorgerà
\\

\textbf{Motivazione}
 & 
Riuscire ad individuare eventuali intrusi da segnalare
\\

\textbf{Influisce}
 & 
Notifiche da segnalare all'utilzzatore
\\

\textbf{Specifica}
 & 
FRS2
\\
\hline
\end{tabular}
\end{figure}

~

\begin{figure}[htbp]
\centering
\begin{tabular}{|c|l|}
\hline

\textbf{Nome}
 & 
Suono Allarme
\\

\textbf{ID}
 & 
FR3
\\

\textbf{Descrizione}
 & 
Se dovesse avvenire un movimento nell'ambiente e nessun utente
qualificato ne darà il consenso, il sistema ROBOT comincerà ad emettere
un suono di allarme
\\

\textbf{Motivazione}
 & 
Scoraggiare eventuali entrusi a compire atti illeciti
\\

\textbf{Influisce}
 & 
Utilizzo autoparlanti Lego per emissione suono allarme
\\

\textbf{Specifica}
 & 

\\
\hline
\end{tabular}
\end{figure}

~

\begin{figure}[htbp]
\centering
\begin{tabular}{|c|l|}
\hline

\textbf{Nome}
 & 
PIN disinnesco
\\

\textbf{ID}
 & 
FR4
\\

\textbf{Descrizione}
 & 
Immissione da parte di un utente qualificato a disattivare le funzioni
d'allarme del ROBOT mediante un sistema di riconoscimento a PIN
\\

\textbf{Motivazione}
 & 
Possibilità, da parte dell'utente, di disattivare il sistema senza
l'utilizzo di uno smartphone
\\

\textbf{Influisce}
 & 
Pressione tasti del sistema ROBOT per cambiare il suo stato da Attivo a
Disattivo
\\

\textbf{Specifica}
 & 
FRS4
\\
\hline
\end{tabular}
\end{figure}

~

\hypertarget{applicazione}{%
\subsubsection{APPLICAZIONE}\label{applicazione}}

\begin{figure}[htbp]
\centering
\begin{tabular}{|c|l|}
\hline

\textbf{Nome}
 & 
Primo Avvio
\\

\textbf{ID}
 & 
FR5
\\

\textbf{Descrizione}
 & 
Dopo aver installato l'applicazione e lanciata, viene chiesto all'utente
di effettuare una primo paired (accoppiamento) con il ROBOT
\\

\textbf{Motivazione}
 & 
Permettere il funzionamento del sistema mediante un accoppiamento con il
ROBOT
\\

\textbf{Influisce}
 & 
Successivi utilizzi dell'applicazione permetteranno il riconoscimento
automatico del ROBOT precedentemente accoppiato
\\

\textbf{Specifica}
 & 
FRS5
\\
\hline
\end{tabular}
\end{figure}

~

\begin{figure}[htbp]
\centering
\begin{tabular}{|c|l|}
\hline

\textbf{Nome}
 & 
Settings' MENU
\\

\textbf{ID}
 & 
FR6
\\

\textbf{Descrizione}
 & 
Eseguendo un TAP sull'apposito campo dei Settings sarà possibile vedere
le informazioni sullo stato del collegamento ed, eventualmente, cambiare
le direttive
\\

\textbf{Motivazione}
 & 
Dare la possibilità all'utente di poter scegliere le preferenze del
comportamente del ROBOT
\\

\textbf{Influisce}
 & 
Visualizzazione del MENU delle Impostazioni
\\

\textbf{Specifica}
 & 
FRS6
\\
\hline
\end{tabular}
\end{figure}

~

\begin{figure}[htbp]
\centering
\begin{tabular}{|c|l|}
\hline

\textbf{Nome}
 & 
Spegnimento e configurazione calendario programmato
\\

\textbf{ID}
 & 
FR7
\\

\textbf{Descrizione}
 & 
Possibilità di spegnere il sistema d'allarme da remoto. Si da, inoltre,
la possibilità all'utente di impostare un calendario per la
programmazione automatica delle accensioni e degli spegnimenti
\\

\textbf{Motivazione}
 & 
Avere pieno controllo sul sistema d'allarme, potendolo bloccare quando
non più necessario
\\

\textbf{Influisce}
 & 
Spegnimento del sistema d'allarme
\\

\textbf{Specifica}
 & 
FRS7
\\
\hline
\end{tabular}
\end{figure}

~

\begin{figure}[htbp]
\centering
\begin{tabular}{|c|l|}
\hline

\textbf{Nome}
 & 
Visualizzazione LOG d'avviso
\\

\textbf{ID}
 & 
FR8
\\

\textbf{Descrizione}
 & 
L'applicazione notificherà eventuali intrusioni mediante un sistema di
LOG a comparsa
\\

\textbf{Motivazione}
 & 
Dare all'utente la consapevolezza di eventuali intromissioni di entità
terze in ambienti dove è stato installato il sistema d'allarme
\\

\textbf{Influisce}
 & 
Notifiche a video su smartphone
\\

\textbf{Specifica}
 & 
FRS8
\\
\hline
\end{tabular}
\end{figure}

~

\begin{figure}[htbp]
\centering
\begin{tabular}{|c|l|}
\hline

\textbf{Nome}
 & 
Spostamento remoto campo visivo del sistema d'allarme
\\

\textbf{ID}
 & 
FR9
\\

\textbf{Descrizione}
 & 
Dare la possibilità all'utente, mediante un'apposita sezione
dell'applicazione, di effettuare uno spostamento remoto del campo visivo
del ROBOT
\\

\textbf{Motivazione}
 & 
Capacità, da parte dell'utente, di avere pieno controllo sulla scelta
della porzione dell'ambiente dove focalizzarsi per le funzioni di
sicurezza
\\

\textbf{Influisce}
 & 
Mediante pressione di tasti si avrà lo spostamento della
telecamera
\\

\textbf{Specifica}
 & 
FRS9
\\
\hline
\end{tabular}
\end{figure}

\hypertarget{definizione-dei-requisiti-non-funzionali}{%
\subsection{3.3 Definizione dei Requisiti non
Funzionali}\label{definizione-dei-requisiti-non-funzionali}}

\hypertarget{introduzione-1}{%
\subsubsection{3.0 Introduzione}\label{introduzione-1}}

(Da cambiare il testo di descrizione)

I requisiti non funzionali sono proprità di comportamento del prodotto.

Si dividono in: - Requisiti di prodotto: specificano come il prodotto
deve comportarsi; - Requisiti di processo: sono i requisiti che sono la
conseguenza di scelte organizzative; - Requisiti esterni: requisiti che
dipendono da fattori esterni all'app, come le leggi.

~

(Da cambiare descrizione tabella) Si utilizza la segeuente tabella
riassuntiva per descivere i requisiti (NFR: not functional requirement):

~

\begin{figure}[htbp]
\centering
\begin{tabular}{|c|l|}
\hline
\textbf{Codice} & Codice identificativo univoco del NFR\\
\textbf{Nome} & Nome assegnato al NFR\\
\textbf{Descrizione} & Descrizione del NFR\\
\textbf{Obiettivo} & Cosa mira a risolvere il NFR\\
\textbf{Dipendenze} & Da cosa viene ``causato'' il NFR\\
\hline
\end{tabular}
\end{figure}

~

\hypertarget{requisiti-di-prodotto}{%
\paragraph{3.3.1 Requisiti di Prodotto}\label{requisiti-di-prodotto}}

(Da rileggere per la questione Sonar)

\begin{figure}[htbp]
\centering
\begin{tabular}{|c|l|}
\hline

\textbf{Codice}
 & 
NFR-1
\\

\textbf{Nome}
 & 
Buone condizioni di luce
\\

\textbf{Descrizione}
 & 
Avere una buona fonte di luce in prossimità dei sensori, così da
permettere un'ottimale analisi da parte del ROBOT dell'ambiente,
permetterà di ottenere una migliore esperienza d'uso per l'utente. Se
questo requisito dovesse mancare il sensore Sonar offrirà comunque le
funzioni basilari al sistema
\\

\textbf{Obiettivo}
 & 
Correta visualizzazione mediante Sensore Visivo
\\

\textbf{Dipendenze}
 & 
Garantire tutte le funzionalità dell'applicazione
\\
\hline
\end{tabular}
\end{figure}

~

\begin{figure}[htbp]
\centering
\begin{tabular}{|c|l|}
\hline

\textbf{Codice}
 & 
NFR-2
\\

\textbf{Nome}
 & 
Buona connessione di rete
\\

\textbf{Descrizione}
 & 
Necessario un collegamento veloce con smartphone e ROBOT per permettere
un'esperienza ottimale in grado di offrire tutte le funzionalità
precedentemente descritte
\\

\textbf{Obiettivo}
 & 
Risolvere eventuali problemi di comunicazioni
\\

\textbf{Dipendenze}
 & 

\\
\hline
\end{tabular}
\end{figure}

\hypertarget{requisiti-di-processo}{%
\paragraph{3.3.2 Requisiti di Processo}\label{requisiti-di-processo}}

\begin{figure}[htbp]
\centering
\begin{tabular}{|c|l|}
\hline

\textbf{Codice}
 & 
NFR-3
\\

\textbf{Nome}
 & 
Sviluppo applicazione mediante Android Studio
\\

\textbf{Descrizione}
 & 
L'applicazione è stata pensata per funzionare su dispositivi Android: si
è quindi deciso di utilizzare l'ambiente di sviluppo integrato Android
Studio, offerto da Google e disponibile gratuitamente sotto licenzo
Apache 2.0
\\

\textbf{Obiettivo}
 & 
Offrire una Companion App pensata come parte integrante del sistema
d'allarme
\\

\textbf{Dipendenze}
 & 
Necessità di un controllo remoto del Sistema d'allarme da parte
dell'utente
\\
\hline
\end{tabular}
\end{figure}

~

\begin{figure}[htbp]
\centering
\begin{tabular}{|c|l|}
\hline

\textbf{Codice}
 & 
NFR-4
\\

\textbf{Nome}
 & 
Sviluppo codice in Java
\\

\textbf{Descrizione}
 & 
L'applicazione, presente su smartphone, verrà scritta e sviluppata in
Java: linguaggio di programmazione conosciuto da tutti i membri del
gruppo
\\

\textbf{Obiettivo}
 & 
Sviluppo applicazione
\\

\textbf{Dipendenze}
 & 

\\
\hline
\end{tabular}
\end{figure}

(AGGIUNGERE REQUISITI DI PROCESSO PER ROBOT)

\hypertarget{requisiti-esterni}{%
\paragraph{3.3.3 Requisiti Esterni}\label{requisiti-esterni}}

\begin{figure}[htbp]
\centering
\begin{tabular}{|c|l|}
\hline

\textbf{Codice}
 & 
NFR-5
\\

\textbf{Nome}
 & 
Privacy
\\

\textbf{Descrizione}
 & 
L'utente dovrà essere a conoscenza ed accettare le richieste
amministrative per l'utilizzo del sistema d'allarme, sia per la parte
relativa all'applicazione smartphone che quella relativa al ROBOT
\\

\textbf{Obiettivo}
 & 
Implementazione di un sistema infomativo con necessità di conferma da
parte dell'utente
\\

\textbf{Dipendenze}
 & 

\\
\hline
\end{tabular}
\end{figure}

\hypertarget{evoluzione-del-sistema}{%
\subsection{3.4 Evoluzione del sistema}\label{evoluzione-del-sistema}}

In futuro, questo sistema potrebbe diventare un ecosistema di
dispositivi che comunicano con l'applicazione in modo da fornire un
controllo da remoto di vari aspetti della casa, come temperatura o
consumi energetici.

\hypertarget{specifica-dei-requisiti}{%
\subsection{3.5 Specifica dei Requisiti}\label{specifica-dei-requisiti}}

\begin{figure}[htbp]
\centering
\begin{tabular}{|c|l|}
\hline
\textbf{ID} & FRS1\\
\textbf{Input} &\\
\textbf{Output} & PIN d'accoppiamento\\
\textbf{Pre-condizioni} & Avere un Robot con il programma
caricato\\
\textbf{Post-condizioni} & Robot pronto per
l'accoppiamento\\
\textbf{Requisiti non funzionali} &\\
\hline
\end{tabular}
\end{figure}

~

\begin{figure}[htbp]
\centering
\begin{tabular}{|c|l|}
\hline

\textbf{ID}
 & \begin{minipage}[t]{0.56\columnwidth}\raggedright
FRS2
\\

\textbf{Input}
 & \begin{minipage}[t]{0.56\columnwidth}\raggedright
Un intruso entra nel campo visivo del Robot
\\

\textbf{Output}
 & \begin{minipage}[t]{0.56\columnwidth}\raggedright
Il Robot emette un suono d'allarme ed invia una notifica
all'utente
\\

\textbf{Pre-condizioni}
 & \begin{minipage}[t]{0.56\columnwidth}\raggedright
L'allarme deve essere innescato
\\

\textbf{Post-condizioni}
 & \begin{minipage}[t]{0.56\columnwidth}\raggedright
Esegue le azioni conseguenti alla scoperta dell'intruso
\\

\textbf{Requisiti non funzionali}
 & \begin{minipage}[t]{0.56\columnwidth}\raggedright

\\
\hline
\end{tabular}
\end{figure}

~

\begin{figure}[htbp]
\centering
\begin{tabular}{|c|l|}
\hline
\textbf{ID} & FRS4\\
\textbf{Input} & Inserimento del PIN di disinnesco\\
\textbf{Output} & Il disinnesco dell'allarme senza dover ricorrere
all'applicazione\\
\textbf{Pre-condizioni} & L'allarme deve essere innescato\\
\textbf{Post-condizioni} &\\
\textbf{Requisiti non funzionali} &\\
\hline
\end{tabular}
\end{figure}

~

\begin{figure}[htbp]
\centering
\begin{tabular}{|c|l|}
\hline
\textbf{ID}
 & 
FRS5
\\
\textbf{Input}
 & 
Inserimento di un PIN di accoppiamento, scelta di un PIN di disinnesco
offline
\\
\textbf{Output}
 & 

\\
\textbf{Pre-condizioni}
 & 
É necessario avere un Robot pronto per l'accoppiamento
\\
\textbf{Post-condizioni}
 & 
L'applicazione puó essere usata
\\
\textbf{Requisiti non funzionali}
 & 

\\
\hline
\end{tabular}
\end{figure}

~

\begin{figure}[htbp]
\centering
\begin{tabular}{|c|l|}
\hline
\textbf{ID} & FRS6\\
\textbf{Input} &\\
\textbf{Output} & Scelta delle impostazioni per il Robot\\
\textbf{Pre-condizioni} & L'applicazione deve essere
operativa\\
\textbf{Post-condizioni} &\\
\textbf{Requisiti non funzionali} &\\
\hline
\end{tabular}
\end{figure}

~

\begin{figure}[htbp]
\centering
\begin{tabular}{|c|l|}
\hline
\textbf{ID} & FRS7\\
\textbf{Input} &\\
\textbf{Output} & Innesco e disinnesco automatico e manuale
dell'allarme\\
\textbf{Pre-condizioni} & L'applicazione deve essere
operativa\\
\textbf{Post-condizioni} & Andarsene via di casa
tranquillo\\
\textbf{Requisiti non funzionali} &\\
\hline
\end{tabular}
\end{figure}

~

\begin{figure}[htbp]
\centering
\begin{tabular}{|c|l|}
\hline
\textbf{ID}
 & 
FRS8
\\
\textbf{Input}
 & 

\\
\textbf{Output}
 & 
Uno storico di tutte le intrusioni avvenute con foto allegate
\\
\textbf{Pre-condizioni}
 & 
L'allarme deve essersi innescato almeno una volta
\\
\textbf{Post-condizioni}
 & 
Puó cancellare intrusioni dalla storico, qualora non le ritenga piú
utili
\\
\textbf{Requisiti non funzionali}
 & 

\\
\hline
\end{tabular}
\end{figure}

~

\begin{figure}[htbp]
\centering
\begin{tabular}{|c|l|}
\hline
\textbf{ID} & FRS9\\
\textbf{Input} & Cliccare sull'apposito pulsante\\
\textbf{Output} & Uno stream di immagini in diretta dal
robot\\
\textbf{Pre-condizioni} & Il Robot deve essere acceso\\
\textbf{Post-condizioni} & Possibilitá di muovere la webcam in tempo
reale\\
\textbf{Requisiti non funzionali} &\\
\hline
\end{tabular}
\end{figure}

\hypertarget{appendici}{%
\section{4 Appendici}\label{appendici}}

\end{document}
